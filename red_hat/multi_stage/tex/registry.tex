\begin{frame}
    \sectiontitleframe
\end{frame}

\begin{frame}
    \autotitle
    Goals
    \begin{itemize}
        \item discoverable
        \item referenceable
        \item verifiable
        \item reusable
    \end{itemize}
    \note{
        Beyond improving the implementation of tests, the design of multi-stage
        tests also had the major goal of improving the \emph{process} of
        creating tests.  It identified several attributes which the new format
        should have, discussed in the next sections.
    }
\end{frame}

\subsection{Discoverable}

\begin{frame}
    \autotitle
    \small
    \begin{itemize}
        \item \url{https://steps.ci.openshift.org}
        \item \url{https://steps.ci.openshift.org/workflow/ipi-aws}
        \item \url{https://steps.ci.openshift.org/chain/ipi-aws-pre}
        \item \url{https://steps.ci.openshift.org/reference/ipi-install-install}
    \end{itemize}
    \note{
        Each component of a test definition is easily discoverable.  There is no
        longer the need to excavate giant \texttt{bash} scripts embedded in YAML
        definitions.  \texttt{steps.ci.openshift.org} is a web interface for
        test definitions which is heavily cross-linked.  All jobs, tests, and
        registry components are listed, in a manner which makes it easy to
        discover exactly how a test is or can be defined.
    }
\end{frame}

\subsection{Referenceable}

\begin{frame}[fragile]
    \autotitle
    \footnotesize
    \url{https://prow.ci.openshift.org/view/gs/origin-ci-test/logs/periodic-ci-openshift-release-master-okd-4.10-e2e-vsphere/1579723667426775040}
    \vspace{\baselineskip}
    \begin{verbatim}
Running step e2e-vsphere-ipi-install-install.
Logs for container test in pod e2e-vsphere-ipi-install-install:
…
Step e2e-vsphere-ipi-install-install failed after 23m20s.
Step phase pre failed after 40m10s.
…
Link to step on registry info site: …
Link to job on registry info site: …
    \end{verbatim}
    \note{
        When a job fails, the \texttt{ci-operator} output includes links to the
        definitions of the steps which failed.
    }
\end{frame}

\begin{frame}
    \autotitle
    \footnotesize
    \url{https://steps.ci.openshift.org/workflow/ipi-aws\#approvers}
    \normalsize
    \begin{itemize}
        \item wking
        \item vrutkovs
        \item abhinavdahiya
        \item deads2k
        \item staebler
        \item technical-release-team-approvers
        \item jianlinliu
        \item yunjiang29
    \end{itemize}
    \note{
        Each of those pages also has links to the definition in GitHub, as well
        as a list of its \texttt{OWNERS} (reviewers and approvers).  This way,
        the error output has (in principle) the information required to go from
        a failure to the possible cause and to those who may be able to assist.
    }
\end{frame}

\subsection{Verifiable}

\begin{frame}[fragile]
    \autotitle
    \footnotesize
    \begin{itemize}
        \item \texttt{pull-ci-openshift-release-master-step-registry-shellcheck}
        \item \url{https://www.shellcheck.net}
        \item
            \begin{verbatim}
find ci-operator/step-registry -name '*.sh' -print0 \
    | xargs -0 -n1 shellcheck -S warning
            \end{verbatim}
    \end{itemize}
    \note{
        The quality of scripts in the step registry is verified using
        \texttt{shellcheck}, a (Haskell) program which identifies problems in
        \texttt{bash} source code (syntax errors, unquoted variables, etc.).  It
        is executed as a blocking pre-submit job which verifies every shell
        script in the registry directory.
    }
\end{frame}

\subsection{Reusable}

\begin{frame}
    \autotitle
    \begin{itemize}
        \item reference
        \item chain
        \item workflow
    \end{itemize}
    \note{
        The step registry is a central place where parts of test definitions are
        stored.  Several types of registry \textit{components} can be combined
        and used by a large number of tests.
    }
\end{frame}

\begin{frame}[fragile]
    \autotitle
    \footnotesize
    \url{https://steps.ci.openshift.org/reference/ipi-install-install}
    \vspace{\baselineskip}
    \begin{verbatim}
ref:
  as: ipi-install-install
  from: installer
  grace_period: 10m
  commands: ipi-install-install-commands.sh
  cli: latest
  resources:
    requests:
      cpu: 1000m
      memory: 2Gi
    \end{verbatim}
    (cont.)
    \note{
        A \textit{reference} is the lowest level of step definition.  It
        corresponds directly to the step definition inline in a test shown in
        previous examples.  In this way, sharing code between tests can be as
        simple as moving a step definition virtually unchanged to the registry
        and referencing it.

        In this example, the installation step which creates ephemeral clusters
        is defined once in the registry and used everywhere it is needed with a
        simple \texttt{ref: ipi-install-install}.
    }
\end{frame}

\begin{frame}[fragile]
    \autotitle
    \footnotesize
    (cont.)
    \begin{verbatim}
  credentials:
  - namespace: test-credentials
    name: loki-stage-collector-test-secret
    mount_path: /var/run/loki-secret
  # …
  env:
  - name: OPENSHIFT_INSTALL_EXPERIMENTAL_DUAL_STACK
    default: "false"
    documentation: Using experimental Azure dual-stack support
  # …
  dependencies:
  - name: "release:latest"
    env: OPENSHIFT_INSTALL_RELEASE_IMAGE_OVERRIDE
  # …
  documentation: |-
    The IPI install step runs the OpenShift Installer …
    \end{verbatim}
\end{frame}

\begin{frame}[fragile]
    \autotitle
    \footnotesize
    \url{https://steps.ci.openshift.org/chain/ipi-aws-pre}
    \vspace{\baselineskip}
    \begin{verbatim}
chain:
  as: ipi-aws-pre
  steps:
  - chain: ipi-conf-aws
  - chain: ipi-install
  documentation: |-
    The IPI setup step contains all steps that provision an
    OpenShift cluster with a default configuration on AWS.
    \end{verbatim}
    \note{
        Sequences of steps can be combined into \textit{chains}, which are
        analogous to the phases of a multi-stage test.  Chains are the main
        method of code reuse in the registry.  Beyond simply grouping steps,
        they can also contain definitions for step parameters and other options,
        as explained later.
    }
\end{frame}

\begin{frame}[fragile]
    \autotitle
    \footnotesize
    \url{https://steps.ci.openshift.org/workflow/ipi-aws}
    \vspace{\baselineskip}
    \begin{verbatim}
workflow:
  as: ipi-aws
  steps:
    pre:
    - chain: ipi-aws-pre
    post:
    - chain: ipi-aws-post
  documentation: |-
    The IPI workflow provides pre- and post- steps that
    provision and deprovision an OpenShift cluster with a
    default configuration on AWS, allowing job authors to
    inject their own end-to-end test logic.

    All modifications to this workflow should be done by
    modifying the `ipi-aws-{pre,post}` chains to allow other
    workflows to mimic and extend this base workflow without
    a need to backport changes.
    \end{verbatim}
    \note{
        Finally, a complete test definition can be grouped into a
        \textit{workflow}.  A workflow definition is exactly the same as a test
        definition, and usually encapsulates a test flow from beginning to end.

        With a workflow, a shared test definition can be as simple as a few
        lines.  If necessary, each phase (i.e. \texttt{pre}, \texttt{test},
        \texttt{post}) can still be redefined in the test which includes the
        workflow.  This will replace the entire sequence of steps for that
        particular phase.
    }
\end{frame}

\begin{frame}[fragile]
    \autotitle
    \begin{multicols}{2}
        \begin{verbatim}
as: e2e-aws
steps:
  pre:
  - as: conf-this
    commands: # …
  - as: conf-that
    commands: # …
  - as: install
    commands: # …
  - as: rbacs
    commands: # …
        \end{verbatim}
        \columnbreak
        \begin{verbatim}
  test:
  - as: test
    commands: # …
  post:
  - as: gather-this
    commands: # …
  - as: gather-that
    commands: # …
  - as: uninstall
    commands: # …
        \end{verbatim}
    \end{multicols}
    \note{
        The next few examples will demonstrate the process of going from a
        completely idiosyncratic test to one that can be shared by multiple
        definitions in a few lines of YAML.

        This definition shows a typical OpenShift E2E test: a cluster is created
        based on some configuration steps, the tests are executed, and the
        cluster is destroyed.  Here, \texttt{commands\ldots} in each definition
        stands for the particular entries which would be declared in each step,
        which can anywhere from several lines to several pages long.
    }
\end{frame}

\begin{frame}[fragile]
    \autotitle
    \begin{multicols}{2}
        \begin{verbatim}
as: e2e-aws
steps:
  pre:
  - ref: conf-this
  - ref: conf-that
  - ref: install
  - ref: rbacs
  # …
        \end{verbatim}
        \columnbreak
        \begin{verbatim}
ref:
  as: conf-this
  commands: # …
        \end{verbatim}
        \begin{verbatim}
ref:
  as: conf-that
  commands: # …
        \end{verbatim}
        \ldots
    \end{multicols}
    \note{
        The first\ldots{} step in the process would be to move individual step
        definitions to the registry and replace the original ones with
        references.  This allows any number of tests to use the shared
        definition and is already an enormous improvement over template tests.
    }
\end{frame}

\begin{frame}[fragile]
    \autotitle
    \begin{multicols}{2}
        \begin{verbatim}
as: e2e-aws
steps:
  pre:
  - chain: aws-pre
  test: # …
  post:
  - chain: aws-post
        \end{verbatim}
        \columnbreak
        \begin{verbatim}
chain:
as: aws-pre
  steps:
  - ref: conf-this
  - ref: conf-that
  - ref: install
  - ref: rbacs
        \end{verbatim}
        \begin{verbatim}
chain:
as: aws-post
steps:
# …
        \end{verbatim}
    \end{multicols}
    \note{
        Next, a sequence of steps from the registry can be placed in one or more
        chains.  This allows the sequence (and/or its configuration) to be
        changed without the need to modify every test definition.
    }
\end{frame}

\begin{frame}[fragile]
    \autotitle
    \begin{multicols}{2}
        \begin{verbatim}
as: e2e-aws
steps:
  workflow: aws-ipi
  test: # …
        \end{verbatim}
        \columnbreak
        \begin{verbatim}
workflow:
  as: aws-ipi
  pre:
  - chain: aws-pre
  post:
  - chain: aws-post
        \end{verbatim}
    \end{multicols}
    \note{
        A complete test pattern (e.g. ``E2E test on AWS'') can then be put into
        a workflow.  This abstracts the setup and cleanup phases so that the
        test definition contains only a reference to the workflow and the actual
        test steps which are to be executed.
    }
\end{frame}

\begin{frame}[fragile]
    \autotitle
    \begin{columns}[t]
        \column{0.4\textwidth}
        \begin{verbatim}
as: e2e-aws
steps:
  workflow: aws-ipi
        \end{verbatim}
        \column{0.55\textwidth}
        \begin{verbatim}
workflow:
  as: openshift-e2e-aws
  pre:
  - chain: aws-pre
  test:
  - ref: openshift-e2e-test
  post:
  - chain: aws-post
        \end{verbatim}
    \end{columns}
    \note{
        Going even further, an entire test suite can be shared among several
        repositories.  In this case, the ``OpenShift on AWS E2E'' test suite is
        put into its own workflow (n.b.: which shares the pre/post chains with
        other workflows).  Repositories can include this entire suite by simply
        declaring a test which references the workflow.
    }
\end{frame}

\begin{frame}[fragile]
    \autotitle
    \footnotesize
    \begin{verbatim}
$ find ci-operator/step-registry/ -name 'ipi-conf-*-ref.yaml' \
    | wc -l
75
$ find ci-operator/step-registry/ -name 'ipi-conf-*-ref.yaml' \
    | sed 's,.*/,,; s/-ref\.yaml//' | shuf | head -15 | sort
ipi-conf-additional-enabled-capabilities
ipi-conf-alibabacloud
ipi-conf-azure-provisioned-des
ipi-conf-azurestack-creds
ipi-conf-azure-vmgenv1
ipi-conf-azure-workers-marketimage
ipi-conf-etcd-on-ramfs
ipi-conf-libvirt
ipi-conf-openstack-enable-octavia
ipi-conf-ovirt-generate-csi-test-manifest
ipi-conf-ovirt-generate-csi-test-manifest-release-4.6-4.8
ipi-conf-ovirt-generate-install-config
ipi-conf-ovirt-generate-install-config-params
ipi-conf-ovirt-generate-ovirt-config
ipi-conf-vsphere-zones
    \end{verbatim}
    \note{
        Since the configuration and installation steps have their own chains in
        the registry, it is very easy to combine one or more configuration steps
        to create a specific test scenario, then include all of the other
        registry components which implement the machinery behind the test
        (cluster installation, artifact gathering, etc.).

        A quick look at the registry shows there is a large number of
        configuration steps, many of which can be combined (e.g. to create a
        test on a cluster with "additional capabilities" and "\texttt{etcd} on
        \texttt{ramfs} using \texttt{libvirt}).
    }
\end{frame}

\begin{frame}[fragile]
    \autotitle
    \footnotesize
    \url{https://docs.ci.openshift.org/docs/architecture/step-registry/#hierarchical-propagation}
    \vspace{\baselineskip}
    \normalsize
    \begin{verbatim}
as: openshift-e2e-test
env:
- name: TEST_SUITE
# …
    \end{verbatim}
    \note{
        One last, more advanced method of reuse is what is called
        \textit{hierarchical propagation}.  Registry components used by a test
        are arranged as a tree: the test can include a workflow, which can
        include any number of chains, which can themselves include chains
        recursively, which ultimately include steps.

        When all of these definitions are assembled to generate the final step
        list executed by the test, several configuration options are propagated
        from the root of the tree to the leaves (namely: parameters,
        dependencies, leases).  This means definitions in tests override those
        in workflows, which in turn override those in chains, which in turn
        override those in steps.  These definitions are strictly checked such
        that every option declared in a parent component has to be declared in a
        subcomponent and has to have a resulting value after this process
        completes.
    }
\end{frame}

\begin{frame}[fragile]
    \autotitle
    \begin{verbatim}
tests:
- as: e2e
  steps:
    test:
    - ref: openshift-e2e-test
    env:
      TEST_SUITE: openshift/conformance/parallel
    \end{verbatim}
    \note{
        In this example, a test includes the step shown in the previous slide
        and specifies a value for its parameter.  It must do so, since the
        parameter does not have a default.  Similarly, giving the parameter a
        value without including a step which declares it would be an error.
    }
\end{frame}

\begin{frame}[fragile]
    \autotitle
    \begin{verbatim}
workflow:
  as: openshift-e2e-serial
  steps:
    test:
    - ref: openshift-e2e-test
    env:
      TEST_SUITE: openshift/conformance/serial
    \end{verbatim}
    \begin{verbatim}
tests:
- as: e2e
  steps:
    workflow: openshift-e2e-serial
    \end{verbatim}
    \note{
        Hierarchical propagation allows the parameter value to be defined in a
        workflow (or a chain).  This way, any test which includes the workflow
        will use that value for the parameter, since it would propagate down
        from the workflow to the step.

        While nonsensical in this case, the test could also give the parameter a
        value, which would override the value in the workflow, since tests are
        the root of the resolution tree.
    }
\end{frame}
