\begin{frame}
    \autotitle
    \begin{itemize}
        \item \url{https://docs.google.com/document/d/1md-1BMf4_7mtKgGVoeZ3jOh4zSIBSjwl6vTTAYESwIM}
            \begin{itemize}
                \item
                    \normalsize
                    \textit{Multi-Stage Tests Design Document}
            \end{itemize}
        \item \url{https://docs.ci.openshift.org}
            \begin{itemize}
                \item \href
                    {https://docs.ci.openshift.org/docs/architecture/step-registry/}
                    {\texttt{docs/architecture/step-registry}}
                \item \href
                    {https://docs.ci.openshift.org/docs/architecture/ci-operator/}
                    {\texttt{docs/architecture/ci-operator}}
            \end{itemize}
    \end{itemize}
    \note{
        Documentation for the topics covered today is somewhat scattered among
        several pages.  The main content is in the dedicated step registry page,
        but some descriptions and examples can also be found in more general
        pages such as \texttt{ci-operator} and others.

        The original design document is also a good source of information about
        the basic architecture.  It also describes very well the historical
        context in which multi-stage tests and the step registry were developed
        and added to \texttt{ci-operator} and the OpenShift CI.
    }
\end{frame}

\subsection{Motivation}

\begin{frame}
    \autotitle
    ca. Aug 2019
    \begin{itemize}
        \item Two test types.
            \begin{itemize}
                \item container
                \item template
            \end{itemize}
        \item Desire to create tests for increasingly varied scenarios.
        \item Existing tests already complex and barely maintained.
    \end{itemize}
    \note{
        That context in summary is this: \texttt{ci-operator} started its life
        supporting only simple \textit{container} tests.  These are fairly
        self-contained tests which execute a single command using a container
        image.

        Then (likely on a Sunday), \textit{template} tests were added.  These
        were amorphous tests which bypassed most of the configuration format and
        instead injected a new test definition at runtime.  Creating and
        maintaining a template test was unnecessarily difficult, so practically
        only a few people had the knowledge and the stomach to do it.

        At the same time, the OpenShift CI was growing, being used both by more
        components and for more varied types of tests.  It was clear that
        requiring a new template test for each new test scenario would be
        impossible, so a new format for test definitions had to be created.
    }
\end{frame}

\begin{frame}
    \autotitle
    Ah, the templates\ldots
    \vfill
    \begin{quote}
complex, esoteric and fragile
    \end{quote}
    \vfill
    \begin{quote}
difficult to extend and use
    \end{quote}
    \vfill
    \begin{quote}
not able to share common test logic
    \end{quote}
    \vfill
    \begin{quote}
duplication and fragmentation
    \end{quote}
    \vfill
    \note{
        This sentiment is visible in the design document, which constantly
        mentions the limitations of templates which impeded the maintenance of
        existing and creation of new tests.
    }
\end{frame}

\begin{frame}
    \autotitle
    \begin{itemize}
        \item Small number of extremely complex \texttt{Pod} definitions.
            \begin{itemize}
                \item
                    Python embedded in Bash embedded in YAML embedded in
                    \ldots
                \item Each responsible for the entire execution of an E2E test.
            \end{itemize}
        \item
            Equally small set of people willing to / capable of ``maintaining''
            them.
        \item Adding a new test scenario
            \begin{itemize}
                \item copying an existing template (thousands of lines of YAML)
                \item minor edits
                \item (extreme duplication)
            \end{itemize}
        \item
            Configuration exposed and required knowledge of byzantine
            implementation details of \texttt{ci-operator}.
        \item<2> \emph{etc.}
    \end{itemize}
    \note{
        Here is a small selection (an entire presentation could be made about
        them):

        \begin{itemize}
            \item
                Template tests are defined in a single YAML file containing,
                among other items, a \texttt{Pod} definition.  The definition
                consisted of several inline \texttt{bash} scripts, sometimes
                with multiple fragments of other languages inside them.
            \item
                The entirety of the test flow had to be contained in this single
                YAML file.
            \item
                Because definitions were completely self-contained, creating new
                types of tests required blindly copying colossal YAML files and
                editing usually just a few lines, creating massive duplication
                and divergence.  Changing common code required editing all
                copies.
            \item
                The integration with other parts of \texttt{ci-operator}
                (images, releases, etc.) was very precarious, requiring test
                authors to know obscure aspects of the underlying
                implementation.
        \end{itemize}

        For these reasons and many others, the set of maintainers of these test
        definitions was virtually non-existent.
    }
\end{frame}
