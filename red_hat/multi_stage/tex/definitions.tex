\begin{frame}
    \sectiontitleframe
\end{frame}

\begin{frame}
    \autotitle
    \begin{itemize}
        \item regular \texttt{ci-operator} test
            \begin{itemize}
                \item images
                \item release images
                \item artifacts
                \item cluster profiles
                \item \ldots
            \end{itemize}
    \end{itemize}
    \note{
        The first thing to note about multi-stage tests is that they are just
        another type of test, just as container tests are.  This
        makes all other parts of \texttt{ci-operator} easily and naturally
        available to them.

        On the implementation side, they are just another type of
        \texttt{ci-operator} step and are fully defined by the configuration
        file (and registry).  Parsing and interpreting a multi-stage test does
        not require any of the arcane techniques used in the implementation of
        template tests.
    }
\end{frame}

\begin{frame}[fragile]
    \autotitle
    \footnotesize
    \url{https://github.com/openshift/release/blob/master/ci-operator/config/openshift/ci-tools/openshift-ci-tools-master.yaml}
    \normalsize
    \begin{verbatim}
tests:
- as: e2e
  steps:
    test:
    - as: e2e
      commands: … make e2e
      from: test-bin
      # …
    \end{verbatim}
    \note{
        Multi-stage tests are defined in the configuration file just like any
        other type of test.  A multi-stage test is distinguished by the
        \texttt{steps} entry.  Definitions can be complex, but \texttt{e2e} in
        \texttt{ci-tools} is a good example of a minimal test.

        (the actual test also has a \texttt{credentials} section, which is not
        revelant for this example and will be explained later)
    }
\end{frame}

\begin{frame}[fragile]
    \autotitle
    \small
    \begin{multicols}{2}
        \begin{verbatim}
tests:
- as: e2e
  steps:
    test:
    - as: e2e
      commands: … make e2e
      from: test-bin
      # …
        \end{verbatim}
        \columnbreak
        \begin{verbatim}
tests:
- as: e2e
  commands: … make e2e
  container:
    from: test-bin
    # …
        \end{verbatim}
    \end{multicols}
    \note{
        With this format, it is equivalent to the \texttt{container} test type
        (unifying their implementation is a long-term goal).  These tests will
        have slightly different executions, but will have the same overall
        effect.
    }
\end{frame}

\subsection{Phases}

\begin{frame}
    \autotitle
    \begin{itemize}
        \item pre/test/post
        \item serial execution
            \begin{itemize}
                \item ``short-circuit'' execution for pre/test
                \item post steps always executed
            \end{itemize}
        \item each step corresponds to a \texttt{Pod}
            \begin{itemize}
                \item shared data can be placed in a special directory
            \end{itemize}
    \end{itemize}
    \note{
        What differentiates multi-stage tests from simple container tests (or
        from templates, for that matter) are the \textit{execution phases}:
        \texttt{pre}, \texttt{test}, and \texttt{post}.  Superficially, they are
        simple sequences of steps (isolated test scripts) executed in serial,
        but each phase has distinct semantic characteristics.
        \begin{description}
            \item[\texttt{pre}]
                is a sequence of preparatory steps.  It should perform the setup
                necessary for the test to be executed (e.g. creating an
                ephemeral cluster).  If any of the steps fail, it is assumed
                that the test cannot continue and the rest of the \texttt{pre}
                steps as well as all of the \texttt{test} steps are not
                executed.
            \item[\texttt{test}]
                is a sequence of one or more steps which execute the actual test
                code.  If any of the steps fail, the rest of the steps are not
                executed.
            \item[\texttt{post}]
                is a sequence of steps which releases any resources acquired by
                the previous phases.  It is always executed and, unlike the
                others, always executed in its entirety, even if some of its
                steps fail.
        \end{description}
    }
\end{frame}

\begin{frame}
    \autotitle
    \texttt{\$SHARED\_DIR}
    \begin{itemize}
        \item Small storage space for inter-step data.
        \item Implemented using a Kubernetes \texttt{Secret}.
        \item Hard 1MB limit, no directories.
        \item
            Completely rewritten by the contents of the directory in the pod
            after the step script is executed.
        \item
            State in the ephemeral cluster can be used for higher-bandwidth
            communication between steps.
        \item \texttt{kubeconfig} is treated especially.
        \item
            Data intended for debugging tests should be placed in the artifacts
            directory.
    \end{itemize}
    \note{
        Unlike template tests:

        \begin{itemize}
            \item Each step is executed in its own isolated pod.
            \item
                Execution is serial: a step is executed only after the previous
                step ends.
            \item
                \texttt{ci-operator} manages the execution of individual steps.
        \end{itemize}

        For data which has to be passed between steps, a small storage space is
        provided, which is mounted in every pod.  It is implemented using
        Kubernetes \texttt{Secret}s, so it has limitations.  However, most tests
        can use external means (such as the ephemeral OpenShift cluster) for
        larger data.  The artifacts directory is also available, just like in
        other types of tests.

        Because \texttt{ci-operator} is optimized for OpenShift E2E tests, some
        files in the shared directory (such as the \texttt{kubeconfig} for
        ephemeral clusters) are treated specially.
    }
\end{frame}

\subsection{Images}

\begin{frame}[fragile]
    \autotitle
    Images
    \begin{itemize}
        \item \texttt{from}
            \begin{itemize}
                \item pipeline images
                    \begin{itemize}
                        \item \texttt{root}, \texttt{src}, \texttt{bin}, \ldots
                        \item \texttt{base\_images}
                        \item \texttt{images}
                    \end{itemize}
                \item ``stable'' images
                    \begin{itemize}
                        \item \texttt{releases}
                        \item \texttt{tag\_specification}
                    \end{itemize}
            \end{itemize}
        \item \texttt{from\_image}
            \begin{itemize}
                \item $\approx \texttt{base\_images}$
                \item
                    \begin{verbatim}
from_image:
  namespace: ocp
  name: upi-installer
  tag: 4.12
                    \end{verbatim}
            \end{itemize}
    \end{itemize}
    \note{
        Unlike container tests, steps can be executed using any
        \texttt{ci-operator} image.  Pipeline images, which are either imported
        or built, can be used, as well as images from release streams or
        payloads.

        A mechanism (\texttt{from\_image}) exists for using imported
        images in shared test definitions.  Here, we will just note its mode of
        operation is equivalent to \texttt{base\_images}; shared definitions
        will be explained in the step registry section.
    }
\end{frame}

\subsection{Credentials}

\begin{frame}[fragile]
    \autotitle
    Credentials
    \begin{itemize}
        \item Vault $\to$ build cluster $\to$ test namespace $\to$ test pod
        \item \texttt{ci-operator} must have access to the source namespace.
        \item
            The \texttt{test-credentials} namespace is pre-configured for
            regular users.
        \item Supplanted old methods.
            \begin{itemize}
                \item \texttt{secret}
                \item \texttt{secrets}
                \item \texttt{--secret-dir}
                \item etc.
            \end{itemize}
        \item
            \begin{verbatim}
credentials:
- namespace: ns
  name: name
  mount_path: /path
            \end{verbatim}
    \end{itemize}
    \note{
        Another major improvement over other test types is the
        \texttt{credentials} section available in steps.  The \texttt{Secret}
        objects listed therein will be imported into the test namespace and
        mounted in the corresponding pods.

        The \texttt{Secret} must already exist and be accessible, but no other
        setup is necessary.  For regular users, the current flow is to create a
        secret collection in Vault and synchronize it to the build clusters
        using special values in the credentials.  A \texttt{test-credentials}
        namespace is preconfigured in each cluster for this purpose.
    }
\end{frame}

\subsection{Parameters}

\begin{frame}[fragile]
    \autotitle
    Parameters
    \begin{itemize}
        \item Key/value data declared in a step.
        \item Ultimately become environmental variables.
        \item Can be overridden (coming soon).
    \end{itemize}
    \small
    \begin{verbatim}
as: openshift-e2e-test
from: tests
commands: openshift-e2e-test-commands.sh
env:
- name: TEST_SUITE
  default: openshift/conformance/parallel
  documentation: |
    The test suite to run.  Use 'openshift-test
    run --help' to list available suites.
# …
    \end{verbatim}
    \note{
        Parameters are a way of generating slight test variations without
        needing a completely new step definition.  Ultimately, they are
        key/value data which become simple environmental variables set in the
        corresponding pods.  One or more steps can declare a parameter,
        optionally with a default value --- all parameters must be declared and
        be given a value in a test definition.

        The hierarchical relation between registry components (explained in a
        later section) allows great freedom in how parameters can be given
        values and how tests can be composed.
    }
\end{frame}

\subsection{Dependencies}

\begin{frame}[fragile]
    \autotitle
    Dependencies
    \begin{itemize}
        \item \texttt{ci-operator} image \textit{pull spec} $\to$ test pod
        \item Establishes images $\to$ test dependency.
        \item
            \begin{verbatim}
as: test-step
dependencies:
- name: pipeline:bin
  env: BIN_IMG
- name: release:4.12
  env: RELEASE_4_12\end{verbatim}
        \item
            \begin{verbatim}
#!/bin/bash
use "$BIN_IMG"
use "$RELEASE_4_12"
        \end{verbatim}
    \end{itemize}
    \note{
        Tests sometimes need to refer to imported or built images, not as their
        execution image, but as a general \textit{pull spec}.  For this, the
        \texttt{dependencies} section can be used.  \texttt{name} references
        either a pipeline or release image: the \textit{pull spec} referring to
        the image in the temporary namespace will be  available to the
        corresponding pod as an environmental variable.

        This also establishes the dependencies between the required
        (\texttt{ci-operator}) steps and the test so that the latter is only
        executed when the images are available.
    }
\end{frame}

\begin{frame}[fragile]
    \autotitle
    \small
    \begin{verbatim}
# openshift-e2e-tests-ref.yaml
dependencies:
- name: "release:latest"
  env: OPENSHIFT_UPGRADE_RELEASE_IMAGE_OVERRIDE
    \end{verbatim}
    \begin{verbatim}
# openshift-e2e-tests-commands.sh
openshift-tests run-upgrade \
    "${TEST_UPGRADE_SUITE}" \
    --to-image \
        "${OPENSHIFT_UPGRADE_RELEASE_IMAGE_OVERRIDE}" \
    --options "${TEST_UPGRADE_OPTIONS-}" \
    --provider "${TEST_PROVIDER}" \
    -o "${ARTIFACT_DIR}/e2e.log" \
    --junit-dir "${ARTIFACT_DIR}/junit"
    \end{verbatim}
    \note{
        One (simplified) example is shown here, where the upgrade test
        references whatever the \texttt{latest} release payload is in a
        particular test namespace.
    }
\end{frame}

\begin{frame}[fragile]
    \autotitle
    Leases
    \begin{itemize}
        \item \texttt{ci-operator} $\to$ Boskos $\to$ test pod
        \item Generalization of implicit lease added by cluster profiles
        \item
            Leased resource name is available to the test script via
            environmental variable.
        \item
            \begin{verbatim}
leases:
- env: OVIRT_UPGRADE_LEASED_RESOURCE
  resource_type: ovirt-upgrade-quota-slice
  count: 42
            \end{verbatim}
    \end{itemize}
    \note{
        \texttt{ci-operator} has an integration with the Boskos leasing server,
        which allows \textit{resources} of limited capacity to be declared.
        \texttt{ci-operator} will request these resources before starting the
        execution of the test, periodically renew them, and release them when
        the test ends.

        Initially (even in the template days, but also currently), this was
        triggered by cluster profiles.  Each profile is declared in
        \texttt{ci-tools} and has a resource type associated with it.

        \texttt{leases} are a generalization of this concept.  Instead of a
        cluster profile, or in addition to it, a test or registry component can
        declare additional resources which it requires before it is allowed to
        start.  The name of the resource is available as an environmental
        variable, which is often used to pass some information to the test step.
    }
\end{frame}

\subsection{etc.}

\begin{frame}
    \autotitle
    \begin{itemize}
        \item best-effort steps
        \item catalogues / optional operators
        \item \texttt{KUBECONFIG}
        \item cluster profiles
        \item \texttt{oc} CLI injection
        \item no \texttt{ServiceAccount} credentials
        \item cluster claims
        \item VPN connection
        \item \ldots
    \end{itemize}
    \note{
        This is an abbreviated list of aspects of multi-stage tests which cannot
        be covered today due to time constraints.  Contrary to template tests,
        most of the implementation is properly documented, so consult the pages
        linked at the beginning of the presentation.
    }
\end{frame}
