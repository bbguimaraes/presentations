\begin{frame}
    \sectiontitleframe
\end{frame}

\subsection{Image mirroring}

\begin{frame}
    \autotitle
    ``CI cycle''
    \begin{enumerate}
        \only<1>{\setcounter{enumi}{-1}}
        \only<2>{\setcounter{enumi}{-2}\item ?}
        \item import images / releases
        \item build images
        \item execute tests
        \item promote images
        \item \texttt{goto 0}
    \end{enumerate}
    \note<1>{
        In the next few sections, we are going to look at what is described as
        the "CI cycle" or "CI loop": the process by which a release stream goes
        from version $x$ to version $x + 1$.

        We start with two preexisting sets of images (more on that later) which
        are imported into the test namespace:
        \begin{itemize}
            \item images from the release the particular component is part of
            \item
                auxiliary images, used in image builds and in the execution of
                tests
        \end{itemize}

        Images are then built and tests are executed to validate them, both by
        themselves and incorporated into the release stream (this is why you
        must have image builds / tests if there is a promotion rule).

        Finally, if all checks are satisfied, the images are "promoted", i.e.
        written to the same release stream which was imported at the beginning,
        thus generating the $x + 1$ release.  Future executions of this and
        other pipelines will use the new set of images.
    }
    \note<2>{
        There remains, however, the question of how this process originates:
        if each pipeline execution is an inductive step, what is the basis?
    }
\end{frame}

\begin{frame}[fragile]
    \autotitle
    \scriptsize
    \begin{tikzpicture}[minimum size = 2em, inner xsep = 1em]
        \node[draw] at (4.5, 0) (promotion) {promotion};
        \draw node[draw, minimum width = 4] at (2.5,  0.75) (images) {images};
        \node[draw, minimum width = 4] at (2.5, -0.75) (tests) {tests};
        \node[draw, minimum width = 2]
            at (0, 0.75) (image_import) {image import};
        \node[draw, minimum width = 2]
            at (0, -0.75) (release_import) {release import};
        \node at (-1.6, -0.75) (releases0) {};
        \node[draw] at (-3.5,  0.75) (app_ci) {\texttt{app.ci}};
        \node[draw] at (-3.5, 0) (cincinnati) {\texttt{cincinnati}};
        \node[draw] at (-3.5, -0.75) (rc) {\texttt{release-controller}};
        \node at (4.5, -1) {test namespace};
        \draw[draw, dashed] (-1.3, -1.375) rectangle ++(7, 2.75);
        \draw[->]
            ($ (app_ci.north east)!0.5!(app_ci.east) $)
            -- ($ (image_import.north west)!0.5!(image_import.west) $);
        \draw
            ($ (app_ci.south east)!0.5!(app_ci.east) $)
            -| (releases0.center);
        \draw (cincinnati) -| (releases0.center);
        \draw (rc.east) -| (releases0.center);
        \draw[->] (releases0.center) -- (release_import);
        \draw[->]
            ($ (image_import.north east)!0.5!(image_import.east) $)
            -- ($ (images.north west)!0.5!(images.west) $);
        \draw[->]
            ($ (image_import.south east)!0.5!(image_import.east) $)
            -- ($ (tests.north west)!0.5!(tests.west) $);
        \draw[->]
            ($ (release_import.north east)!0.5!(release_import.east) $)
            -- ($ (images.south west)!0.5!(images.west) $);
        \draw[->]
            ($ (release_import.south east)!0.5!(release_import.east) $)
            -- ($ (tests.south west)!0.5!(tests.west) $);
        \draw[->] (images) -- ($ (promotion.north west)!0.5!(promotion.west) $);
        \draw[->] (tests) -- ($ (promotion.south west)!0.5!(promotion.west) $);
        \draw[->] (promotion) -- ++(0, 1.75) -| (app_ci);
    \end{tikzpicture}
    \note{
        This is the pictorial representation of this process.  Images come from
        the left: base images come from the central registry in \texttt{app.ci}
        (more on that also later), release images come from any of the three
        places, depending on which type of \texttt{releases} field is used.

        The sub-graph which originates in \texttt{app.ci} and returns to it
        finally after the promotion step is the CI cycle.
    }
\end{frame}

\begin{frame}
    \autotitle
    \begin{itemize}
        \item supplemental images
            \begin{itemize}
                \item \url{https://github.com/openshift/release/tree/master/clusters/app.ci/supplemental-ci-images}
                \item \texttt{BuildConfig}s
                \item<2> \texttt{registry.ci.openshift.org/ci/ci-tools-build-root}
            \end{itemize}
        \item image mirroring
            \begin{itemize}
                \item \url{https://github.com/openshift/release/tree/master/core-services/image-mirroring}
                \item Quay/etc. $\leftrightarrow$ \texttt{app.ci}
                \item<2> \texttt{registry.ci.openshift.org/coreos/stream9:9}
            \end{itemize}
        \item ART / OCP builder images
            \begin{itemize}
                \item {\footnotesize\url{https://docs.ci.openshift.org/docs/architecture/images/}}
                \item \url{https://github.com/openshift/ocp-build-data.git}
                \item<2> \texttt{registry.ci.openshift.org/ocp/builder:\ldots}
            \end{itemize}
    \end{itemize}
    \note<1>{
        Base images can come from several places:
        \begin{itemize}
            \item
                Images can be built directly using OpenShift
                \texttt{BuildConfig}s.
            \item
                A mirroring process exists between \texttt{app.ci} and Quay.  It
                is actually bidirectional, but here we are only interested in
                images which are imported from Quay.
            \item
                \textit{Productized} images, used to build official OpenShift
                release images, come from ART.
        \end{itemize}
    }
    \note<2>{
        Note, however, that these images are all located in the \texttt{app.ci}
        central registry.  Initially, they were simply referenced directly, but
        that very soon turned out to not scale to the number of jobs we had at
        the time (which was a small fraction of the current number).
    }
\end{frame}

\subsection{\texttt{dptp-controller-manager}}

\begin{frame}
    \autotitle
    \ttfamily
    \texttt{dptp-controller-manager}
    \begin{itemize}
        \item \href
            {https://github.com/openshift/ci-tools/blob/master/cmd/dptp-controller-manager/}
            {cmd/dptp-controller-manager/}
        \item \href
            {https://github.com/openshift/ci-tools/blob/master/pkg/controller/test-images-distributor/}
            {pkg/controller/test-images-distributor/}
    \end{itemize}
    \note{
        For this reason, there is now a process which imports those images to
        each build cluster whenever required.  It is one of the processes
        executed as part of the \texttt{dptp-controller-manager} (famed cluster
        node assassin) and is named \texttt{test-images-distributor}.
    }
\end{frame}

\begin{frame}[fragile]
    \autotitle
    \small
    \begin{verbatim}
args:
…
- --enable-controller=test_images_distributor
- --enable-controller=promotionreconciler
- --enable-controller=serviceaccount_secret_refresher
- --enable-controller=testimagestreamimportcleaner
…
    \end{verbatim}
    \note{
        The command line shows the enabled controllers, which perform various
        functions in the CI clusters.
    }
\end{frame}

\begin{frame}[fragile]
    \autotitle
    \small
    \begin{verbatim}
…
- --release-repo-git-sync-path=/var/repo/release
- --kubeconfig-dir=/var/kubeconfigs
- --registry-cluster-name=app.ci
- --testImagesDistributorOptions \
    .additional-image-stream-tag=ocp/builder:golang-1.10
…
- --testImagesDistributorOptions \
    .additional-image-stream-tag= \
    ocp/builder:rhel-7-golang-1.11
…
- --testImagesDistributorOptions \
    .additional-image-stream-namespace=ci
- --testImagesDistributorOptions \
    .additional-image-stream=rhcos/machine-os-content
…
    \end{verbatim}
    \note{
        It has a few options to explicitly include image streams and tags\ldots
    }
\end{frame}

\begin{frame}
    \autotitle
    \href
        {https://github.com/openshift/ci-tools/blob/master/pkg/api/helper/imageextraction.go}
        {pkg/api/helper/imageextraction.go}
    \begin{itemize}
        \item \texttt{TestInputImageStreamsFromResolvedConfig}
        \item \texttt{TestInputImageStreamTagsFromResolvedConfig}
    \end{itemize}
    \note{
        \dots{} but its main function is to inspect every \texttt{ci-operator}
        configuration file and extract input images to be synchronized, which it
        does automatically whenever the source streams are changed.
    }
\end{frame}

\subsection{Image promotion}

\begin{frame}[fragile]
    \autotitle
    \small
    \url{https://prow.ci.openshift.org/view/gs/origin-ci-test/logs/branch-ci-openshift-ci-tools-master-images/1561993456950185984}
    \normalsize
    \begin{verbatim}
…
Building src
Build src succeeded after 4m48s
Building bin
Build bin succeeded after 25m54s
Building determinize-peribolos
Building ci-secret-generator
Building ci-operator-config-mirror
…
    \end{verbatim}
    \note{
        Promotion is a relatively simple matter now that we have looked at the
        rest of the pipeline.  We start by building all images not explicitly
        excluded, \ldots
    }
\end{frame}

\begin{frame}[fragile]
    \autotitle
    \begin{verbatim}
…
Build prcreator succeeded after 14m26s
Tagging prcreator into stable
Build private-prow-configs-mirror \
    succeeded after 15m51s
Tagging private-prow-configs-mirror into stable
Promoting tags to ci/${component}:latest: \
    applyconfig, auto-aggregator-job-names, \
    auto-config-brancher, auto-peribolos-sync, \
    auto-sippy-config-generator, …
Ran for 1h7m10s
    \end{verbatim}
    \note{
        \ldots{} then tag them in the \texttt{stable} \texttt{ImageStream} as
        usual, and finally transfer them to the central registry in
        \texttt{app.ci}, where they will be available to future jobs.
    }
\end{frame}
