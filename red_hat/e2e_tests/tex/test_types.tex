\begin{frame}
    \sectiontitleframe
    \note{(we have fancy section title slides now)}
\end{frame}

\begin{frame}[fragile]
    \autotitle
    \href
        {https://github.com/openshift/release/blob/master/ci-operator/config/openshift/origin/openshift-origin-master.yaml}
        {ibid}
    \begin{verbatim}
- as: verify-deps
  commands: make verify-deps …
  container:
    from: src
    \end{verbatim}
    \href
        {https://github.com/openshift/release/blob/master/ci-operator/config/openshift/origin/openshift-origin-release-3.11.yaml}
        {\ttfamily\ldots/openshift-origin-release-3.11.yaml}
    \begin{verbatim}
- as: e2e-gcp
  commands: … run-tests
  openshift_ansible:
    cluster_profile: gcp
    \end{verbatim}
    \note{
        The \texttt{test} entries in the configuration file have many different
        forms, although they have many similarities.  \texttt{steps} (or,
        sometimes, \texttt{literal\_steps}, as in the previous example), denotes
        a \textit{multi-stage} test.  Other basic test types are:
        \begin{itemize}
            \item
                simple \textit{container} tests, declared with
                \texttt{container}
            \item
                a large variety of \textit{template} tests, declared with fields
                in the form \texttt{openshift\_*}
        \end{itemize}
    }
\end{frame}

\begin{frame}[fragile]
    \autotitle
    \url{https://github.com/openshift/ci-tools/blob/master/pkg/api/types.go}
    \vfill
    \footnotesize
    \begin{verbatim}
type TestStepConfiguration struct {
    As string ` json:"as"`
    Commands string ` json:"commands,omitempty"`
    // …
    // Only one of the following can be not-null.
    ContainerTestConfiguration                                …
    MultiStageTestConfiguration                               …
    MultiStageTestConfigurationLiteral                        …
    OpenshiftAnsibleClusterTestConfiguration                  …
    OpenshiftAnsibleSrcClusterTestConfiguration               …
    OpenshiftAnsibleCustomClusterTestConfiguration            …
    OpenshiftInstallerClusterTestConfiguration                …
    OpenshiftInstallerUPIClusterTestConfiguration             …
    OpenshiftInstallerUPISrcClusterTestConfiguration          …
    OpenshiftInstallerCustomTestImageClusterTestConfiguration …
}
    \end{verbatim}
    \vfill
    \note{
        This is manifested in code in the \texttt{TestStepConfiguration}
        structure (not to be confused with the \texttt{TestStep} structure, used
        in multi-stage tests), which uses the common pattern of many (optional)
        pointers to other structures, only one of which is ever non-null (a
        \textit{sum type}).
    }
\end{frame}

\subsection{Container}

\begin{frame}[fragile]
    \autotitle
    \begin{verbatim}
// Only one of the following can be not-null.
ContainerTestConfiguration \
    *ContainerTestConfiguration \
    ` json:"container,omitempty"`
// …
    \end{verbatim}
    \note{
        (these identifiers are enormous, so here is what a full line looks like)
    }
\end{frame}

\begin{frame}[fragile]
    \autotitle
    \begin{verbatim}
type ContainerTestConfiguration struct {
    From PipelineImageStreamTagReference
    MemoryBackedVolume *MemoryBackedVolume
    Clone *bool
}
    \end{verbatim}
    \note{
        Starting with container tests, their structure is deceptively simple.
        It declares its container image plus a couple of other, more esoteric
        fields.
    }
\end{frame}

\begin{frame}[fragile]
    \autotitle
    \begin{verbatim}
type TestStepConfiguration struct {
    As string
    Commands string
    Cluster Cluster
    Secret *Secret
    Secrets []*Secret
    Cron *string
    Interval *string
    ReleaseController bool
    Postsubmit bool
    ClusterClaim *ClusterClaim
    RunIfChanged string
    Optional bool
    SkipIfOnlyChanged string
    Timeout *prowv1.Duration
    // …
    \end{verbatim}
    \note{
        This is because most of the fields live in the original structure,
        previously abbreviated.  The list of fields here is somewhat unruly.  In
        the past, we had a very relaxed policy for external contributions, so
        the code base --- and this area in particular --- grew very
        ``organically'' (to put it favorably).

        Some of these, such as the build cluster, the periodic/post-submit
        fields, etc. are still useful.  Some are obsolete and kept for
        compatibility.

        As an aside, the capabilities of container tests are roughly a subset of
        those of multi-stage, there is a long-term plan to unify their
        underlying implementation.
    }
\end{frame}

\subsection{Template}

\begin{frame}[fragile]
    \autotitle
    \begin{verbatim}
- as: e2e-gcp
  commands: … run-tests
  openshift_ansible:
    cluster_profile: gcp
    \end{verbatim}
    \footnotesize
    \begin{verbatim}
args:
- --image-import-pull-secret=/etc/pull-secret/.dockerconfigjson
- --report-credentials-file=/etc/report/credentials
- --secret-dir=/usr/local/e2e-gcp-periodic-cluster-profile
- --target=e2e-gcp-periodic
- --template=/usr/local/e2e-gcp-periodic
- --gcs-upload-secret=/secrets/gcs/service-account.json
command:
- ci-operator
    \end{verbatim}
    \note{
        The second type of test (also in chronological order) is everyone's
        favorite: template tests.  This was the first mechanism added to
        \texttt{ci-operator} to support end-to-end tests, or in general anything
        more complex than a container test.

        They are mostly a historical curiosity at this point, used only in very
        old, 3.11 jobs, but they provide some context to some of the more
        dubious aspects of \texttt{ci-operator}.

        There is no (with one exception due to a failed plan) corresponding test
        definition in the configuration file for these tests: the entry in
        \texttt{tests} is used exclusively by \texttt{prowgen}.  Instead, the
        definition is supplied at runtime via the \texttt{--template} argument.
    }
\end{frame}

\begin{frame}[fragile]
    \autotitle
    \begin{verbatim}
      volumeMounts:
      - mountPath: /usr/local/e2e-gcp-periodic
        name: job-definition
        subPath: cluster-launch-e2e.yaml
    volumes:
    - configMap:
        name: prow-job-cluster-launch-e2e
      name: job-definition
    \end{verbatim}
    \note{
        In our Prow jobs, this is done by mounting the definition via a
        \texttt{ConfigMap}\ldots
    }
\end{frame}

\begin{frame}[fragile]
    \autotitle
    \url{https://github.com/openshift/release/tree/master/ci-operator/templates}
    \vfill
    \footnotesize
    \begin{verbatim}
ci-operator/templates/
    master-sidecar-3.yaml
    master-sidecar-4.4.yaml
    openshift/
        installer/
            cluster-launch-installer-custom-test-image.yaml
            cluster-launch-installer-e2e.yaml
            cluster-launch-installer-libvirt-e2e.yaml
            cluster-launch-installer-metal-e2e.yaml
            cluster-launch-installer-openstack-e2e.yaml
            cluster-launch-installer-openstack-upi-e2e.yaml
            cluster-launch-installer-src.yaml
            cluster-launch-installer-upi-e2e.yaml
        openshift-ansible/
            cluster-launch-e2e-openshift-ansible.yaml
            cluster-launch-e2e.yaml
            cluster-scaleup-e2e-40.yaml
    \end{verbatim}
    \vfill
    \note{
        \ldots which in turn are populated via \texttt{updateconfig} from the
        files in the dreaded \texttt{ci-operator/templates} directory in
        \texttt{openshift/release}.
    }
\end{frame}

\begin{frame}[fragile]
    \autotitle
    \href
        {https://github.com/openshift/release/blob/master/ci-operator/templates/openshift/installer/cluster-launch-installer-e2e.yaml}
        {\ldots/openshift/installer/cluster-launch-installer-e2e.yaml}
    \vfill
    \begin{verbatim}
kind: Template
apiVersion: template.openshift.io/v1

parameters:
- name: JOB_NAME
  required: true
- name: JOB_NAME_SAFE
  required: true
# …
    \end{verbatim}
    \vfill
    \note{
        Each of these files is an OpenShift \texttt{Template} object, which
        consists of a list of parameters (strings, essentially)\ldots
    }
\end{frame}

\begin{frame}[fragile]
    \autotitle
    \begin{verbatim}
objects:
# We want the cluster to be able to access
# these images
- kind: RoleBinding
  apiVersion: authorization.openshift.io/v1
  metadata:
    name: ${JOB_NAME_SAFE}-image-puller
    namespace: ${NAMESPACE}
  # …
    \end{verbatim}
    \note{
        \ldots and a list of objects.  \texttt{\$\{\ldots\}} strings are
        replaced by parameter values when the template is instantiated (and good
        luck telling what is \texttt{bash} interpolation and what is template
        substitution in a complex \texttt{Pod} definition).
    }
\end{frame}

\begin{frame}[fragile]
    \autotitle
    \begin{verbatim}
# The e2e pod spins up a cluster, runs e2e tests,
# and then cleans up the cluster.
- kind: Pod
  apiVersion: v1
  metadata:
    name: ${JOB_NAME_SAFE}
    namespace: ${NAMESPACE}
  # …
    \end{verbatim}
    \note{
        And that is a summary of the entirety of the capabilities provided by
        template tests.  From there, users would create a \texttt{Pod}
        definition (n.b.: a single one) to execute their test using colossal,
        inline shell scripts.
    }
\end{frame}

\begin{frame}[fragile]
    \autotitle
    \begin{verbatim}
containers:
# Once the cluster is up, executes shared tests
- name: test
# …
# Runs an install
- name: setup
# …
# Performs cleanup of all created resources
- name: teardown
# …
    \end{verbatim}
    \note{
        In practice, a few templates were developed and used by most tests, all
        following roughly this structure, later mirrored in multi-stage tests: a
        \texttt{setup} container performed the cluster installation, a
        \texttt{test} container executed OpenShift or repository tests, and a
        \texttt{teardown} container destroyed the temporary cluster.
    }
\end{frame}

\begin{frame}[fragile]
    \autotitle
    \begin{verbatim}
parameters:
# …
- name: IMAGE_FORMAT
- name: IMAGE_INSTALLER
  required: true
- name: IMAGE_TESTS
  required: true
# …
- name: RELEASE_IMAGE_LATEST
  required: true
# …
    \end{verbatim}
    \note{
        Configuration and parameterization was done via these template
        parameters, some of which are treated especially by
        \texttt{ci-operator}:
        \begin{itemize}
            \item
                \texttt{IMAGE\_FORMAT} is populated with the public registry
                \textit{pull spec} for built images.
            \item
                \texttt{IMAGE\_*} entries are populated with entries from the
                input release stream.
            \item
                \texttt{RELEASE\_IMAGE\_*} entries are populated with the
                release payload \textit{pull spec}.
        \end{itemize}
        The presence of each of these variables also causes the template step to
        depend on the step which provides it (the \texttt{Provides} method in
        each step type).  Environment variables can also be used to initialize
        or override these values, which is still used in some of our E2E tests,
        even in multi-stage (e.g. the release controller uses
        \texttt{RELEASE\_IMAGE\_LATEST} to override the input release payload).
    }
\end{frame}

\subsection{Multi-stage}

\begin{frame}
    \autotitle
    \begin{multicols}{2}
        \begin{itemize}
            \item test definition
            \item test phases
                \begin{itemize}
                    \item pre
                    \item test
                    \item post
                \end{itemize}
            \item step registry
                \begin{itemize}
                    \item references
                    \item chains
                    \item workflows
                    \item observers
                \end{itemize}
            \item container image
            \item parameters
            \item dependencies
            \item credentials
            \item leases
            \item overriding
            \item \ldots
        \end{itemize}
    \end{multicols}
    \note{
        Of course, multi-stage tests are a universe of their own and worth (at
        least) a presentation in themselves.  Here are some of the capabilities,
        most of which we will not have time to analyze today.
    }
\end{frame}

\begin{frame}[fragile]
    \autotitle
    \begin{verbatim}
type MultiStageTestConfiguration struct {
    ClusterProfile ClusterProfile
    Pre []TestStep
    Test []TestStep
    Post []TestStep
    Workflow *string
    Environment TestEnvironment
    Dependencies TestDependencies
    DNSConfig *StepDNSConfig
    Leases []StepLease
    AllowSkipOnSuccess *bool
    AllowBestEffortPostSteps *bool
    Observers *Observers
    DependencyOverrides DependencyOverrides
}
    \end{verbatim}
    \note{
        Two structures, which share most of their fields, are involved in the
        configuration of multi-stage tests.
        \texttt{MultiStageTestConfiguration} is loaded directly from the
        \texttt{steps} field.  It represents a user test definition which
        potentially needs to go through \textit{resolution}, where references to
        steps, chains, and workflows from the step registry have to be replaced
        with their defintions.

        The \texttt{--unresolved-config} argument and the
        \texttt{UNRESOLVED\_CONFIG} variables correspond to this structure.
    }
\end{frame}

\begin{frame}[fragile]
    \autotitle
    \begin{verbatim}
type MultiStageTestConfigurationLiteral struct {
    ClusterProfile ClusterProfile
    Pre []LiteralTestStep
    Test []LiteralTestStep
    Post []LiteralTestStep
    Environment TestEnvironment
    Dependencies TestDependencies
    DNSConfig *StepDNSConfig
    Leases []StepLease
    AllowSkipOnSuccess *bool
    AllowBestEffortPostSteps *bool
    Observers []Observer
    DependencyOverrides DependencyOverrides
    Timeout *prowv1.Duration
}
    \end{verbatim}
    \note{
        Its counterpart is \texttt{MultiStageTestConfigurationLiteral}, which
        represents a \emph{resolved} configuration, and corresponds to the
        \texttt{--config} argument and the \texttt{CONFIG\_SPEC} variable.
    }
\end{frame}

\begin{frame}[fragile]
    \autotitle
    \begin{verbatim}
type LiteralTestStep struct {
    As string
    From string
    FromImage *ImageStreamTagReference
    Commands string
    Resources ResourceRequirements
    Timeout *prowv1.Duration
    GracePeriod *prowv1.Duration
    Credentials []CredentialReference
    Environment []StepParameter
    Dependencies []StepDependency
    \end{verbatim}
    \note{
        This distinction is also reflected in the \texttt{LiteralTestStep}
        structure, lists of which compose the input configuration\ldots
    }
\end{frame}

\begin{frame}[fragile]
    \autotitle
    \begin{verbatim}
    DNSConfig *StepDNSConfig
    Leases []StepLease
    OptionalOnSuccess *bool
    BestEffort *bool
    Cli string
    Observers []string
    RunAsScript *bool
}
    \end{verbatim}
\end{frame}

\begin{frame}[fragile]
    \autotitle
    \begin{verbatim}
type TestStep struct {
    *LiteralTestStep
    Reference *string
    Chain *string
}
    \end{verbatim}
    \note{
        \ldots and the \texttt{TestStep} structure, which has the same contents
        but has also the option of referring to an external definition from the
        registry.
    }
\end{frame}

\begin{frame}[fragile]
    \autotitle
    \begin{verbatim}
tests:
- as: multi-stage
  steps: # …
- as: multi-stage-literal
  literal_steps: # …

$ JOB_SPEC=… ci-operator
$ ci-operator --config …
$ ci-operator --unresolved-config …
$ CONFIG_SPEC=… ci-operator …
$ UNRESOLVED_CONFIG=… ci-operator …
    \end{verbatim}
    \note{
        These two types exist to distinguish the two states in code and between
        services, e.g.:
        \begin{itemize}
            \item
                regular jobs receive a literal configuration from the resolver
            \item
                \texttt{pj-rehearse} loads the unresolved configuration and
                expands it itself based on the PR contents, setting
                \texttt{\$UNRESOLVED\_CONFIG}
            \item
                release jobs provide their own inline configuration via
                \texttt{\$CONFIG\_SPEC} or \texttt{\$UNRSOLVED\_CONFIG}
                depending on the case
            \item etc.
        \end{itemize}
    }
\end{frame}
