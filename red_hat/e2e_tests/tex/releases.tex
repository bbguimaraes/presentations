\begin{frame}
    \sectiontitleframe
\end{frame}

\begin{frame}[fragile]
    \autotitle
    \url{https://github.com/openshift/ci-tools/blob/master/pkg/api/types.go}
    \vfill
    \begin{verbatim}
type ReleaseBuildConfiguration struct {
    Metadata Metadata
    InputConfiguration
    // …
}

type InputConfiguration struct {
    // …
    Releases map[string]UnresolvedRelease
}
    \end{verbatim}
    \vfill
    \note{
        The first major input to E2E tests, seen at the beginning of the output
        log, are the release streams / payloads.  They are configured in the
        \texttt{releases} entry in the configuration file.

        Originally, they were specified in \texttt{tag\_specification}, which
        provides a fixed subset of the same functionality.  That field is now
        deprecated and will be removed, but can still be found in many
        configuration files.
    }
\end{frame}

\begin{frame}[fragile]
    \autotitle
    \begin{verbatim}
type UnresolvedRelease struct {
    // Integration describes an integration stream
    // which we can create a payload out of
    Integration *Integration
    // Candidate describes a candidate release
    // payload
    Candidate *Candidate
    // Prerelease describes a yet-to-be released
    // payload
    Prerelease *Prerelease
    // Release describes a released payload
    Release *Release
}
    \end{verbatim}
    \note{
        The top level keys of the \texttt{releases} field are simply
        identifiers.  Each value is a structure in the familiar format where
        only one of the pointers is ever non-null.
    }
\end{frame}

\begin{frame}[fragile]
    \autotitle
    \begin{verbatim}
type Candidate struct {
    Product ReleaseProduct
    Architecture ReleaseArchitecture
    Stream ReleaseStream
    Version string
    Relative int
}

type Prerelease struct {
    Product ReleaseProduct
    Architecture ReleaseArchitecture
    VersionBounds VersionBounds
}
    \end{verbatim}
    \note{
        The \texttt{release}, \texttt{prerelease}, and \texttt{candidate} types
        all refer to existing payloads: they vary only in their source.
        \texttt{integration} streams (when not overridden, as described later)
        use \texttt{ImageStream}s.
    }
\end{frame}

\begin{frame}[fragile]
    \autotitle
    \begin{verbatim}
pkg/release/
    candidate/
        client.go
        types.go
    client.go
    config/
        client.go
        config.go
    official/
        client.go
        types.go
    prerelease/
        client.go
    \end{verbatim}
    \note{
        The different sources used for these types can be seen in
        \texttt{pkg/release} in \texttt{openshift/ci-tools}.
    }
\end{frame}

\begin{frame}
    \autotitle
    \begin{itemize}
        \item \texttt{candidate} / \texttt{prerelease}
            \begin{itemize}
                \item \url{https://amd64.ocp.releases.ci.openshift.org}
                \item release controller
            \end{itemize}
        \item \texttt{release}
            \begin{itemize}
                \item {\footnotesize\url{https://api.openshift.com/api/upgrades_info/v1/graph}}
                \item cincinnati
            \end{itemize}
    \end{itemize}
    \note{
        Both \texttt{candidate} and \texttt{prerelease} types obtain their
        release payloads from the \texttt{release-controller}, according to the
        input values.

        The \texttt{release} type obtains official releases from
        \texttt{cincinnati}.
    }
\end{frame}

\begin{frame}[fragile]
    \autotitle
    \setlength{\columnsep}{0pt}
    \begin{multicols}{2}
        \vspace*{\fill}
        \footnotesize
        \begin{verbatim}
releases:
  initial:
    integration:
      name: "4.10"
      namespace: ocp
  latest:
    integration:
      include_built_images: \
        true
      name: "4.10"
      namespace: ocp
        \end{verbatim}
        \vfill
        \columnbreak
        \normalsize
        \begin{itemize}
            \item \texttt{ReleaseImagesTagStep}
                \begin{itemize}
                    \item source $\to$ destination \texttt{ImageStream}
                    \item
                        \texttt{\$namespace/\$name} $\to$
                        \texttt{ci-op-*/stable*}
                \end{itemize}
            \item \texttt{AssembleReleaseStep}
                \begin{itemize}
                    \item
                        \texttt{ImageStream} $\to$ \\
                        release payload
                    \item \texttt{stable*} $\to$ \texttt{release:*}
                    \item
                        will wait for built images if
                        \texttt{include\_built\_images}
                \end{itemize}
        \end{itemize}
    \end{multicols}
    \note{
        The two categories (payload vs. stream) determine the steps
        \texttt{ci-operator} will take to import and process the release in
        order to make it available to the test.  \texttt{integration} streams,
        as mentioned previously, come from an \texttt{ImageStream}.  This means
        two steps are required:
        \begin{itemize}
            \item
                \texttt{ReleaseImagesTagStep} will import (i.e. copy) the tags
                from the source.
            \item
                \texttt{AssembleReleaseStep} will create a release payload from
                the resulting \texttt{ImageStream}.  If an entry declares
                \texttt{include\_built\_images}, this will cause this step to
                wait for all images to be built and tagged into the stream,
                so that they can be included in the payload.  This is usually
                the case for \texttt{latest} payloads, so that they can be used
                to test the resulting release containing images built using the
                code under test.
        \end{itemize}
    }
\end{frame}

\begin{frame}[fragile]
    \autotitle
    \setlength{\columnsep}{0pt}
    \begin{multicols}{2}
        \vspace*{\fill}
        \begin{verbatim}
tag_specification:
  namespace: ocp
  name: "4.10"
        \end{verbatim}
        \vfill
        \columnbreak
        \begin{itemize}
            \item always \texttt{initial} and \texttt{latest}
            \item \texttt{include\_built\_images} implicitly for \texttt{latest}
            \item
                \texttt{ReleaseImagesTagStep} ($\approx$
                \texttt{ReleaseSnapshotStep})
            \item \texttt{RELEASE\_IMAGE\_*}
        \end{itemize}
    \end{multicols}
    \note{
        \texttt{tag\_specification} is the precursor to \texttt{integration}
        (and \texttt{releases} in general).  It is legacy now but can be found
        in some old jobs (and sometimes causes problems).  It works roughly like
        a group of fixed values for integration streams.

        Both \texttt{integration} releases and \texttt{tag\_specification} can
        have their values overridden by \texttt{RELEASE\_IMAGE\_*} environment
        variables.  When this happens (e.g. in jobs created by the release
        controller), the images are treated as input release payloads and
        processed as described below.
    }
\end{frame}

\begin{frame}[fragile]
    \autotitle
    \footnotesize
    \begin{verbatim}
$ oc adm release extract \
    --file image-references \
    quay.io/openshift/okd:4.10.0-0.okd-2022-07-09-073606 \
    | yaml
kind: ImageStream
apiVersion: image.openshift.io/v1
metadata:
  name: 4.10.0-0.okd-2022-07-09-073606
  creationTimestamp: 2022-07-10T09:12:53Z
  annotations:
    release.openshift.io/from-image-stream: >
      origin/4.10-2022-07-09-073606
    release.openshift.io/from-release: >
      registry.ci.openshift.org/origin/release:4.10.0-0.…
…
    \end{verbatim}
    \note{
        Here is an example of the relevant contents of a release payload image.
        It contains the name, date of creation, source, \ldots
    }
\end{frame}

\begin{frame}[fragile]
    \autotitle
    \footnotesize
    \begin{verbatim}
spec:
  lookupPolicy:
    local: false
  tags:
  - name: alibaba-cloud-controller-manager
    annotations:
      io.openshift.build.commit.id: 0
      io.openshift.build.commit.ref: release-4.10
      io.openshift.build.source-location: >
        https://github.com/openshift/…
    from:
      kind: DockerImage
      name: quay.io/openshift/okd-content@sha256:…
    generation: null
    importPolicy:
    referencePolicy:
      type: 0
…
    \end{verbatim}
    \note{
        \ldots and the list of image references which will be used in the
        cluster installation and configuration.
    }
\end{frame}

\begin{frame}[fragile]
    \autotitle
    \setlength{\columnsep}{0pt}
    \begin{multicols}{2}
        \small
        \begin{verbatim}
releases:
  latest:
    release:
       architecture: amd64
       channel: stable
       version: "4.10"
        \end{verbatim}
        \columnbreak
        \normalsize
        \begin{itemize}
            \item \texttt{candidate} / \texttt{prerelease}
            \item \texttt{ImportReleaseStep}
                \begin{itemize}
                    \item release payload $\to$ \texttt{ImageStream}
                    \item \texttt{\$src} $\to$ \texttt{release:*}
                    \item tags $\to$ \texttt{ImageStream}
                    \item
                        \texttt{release:*} $\to$ \\
                        \texttt{oc \ldots{} extract} $\to$ \\
                        \texttt{stable*}
                \end{itemize}
        \end{itemize}
    \end{multicols}
    \note{
        The other types of releases use a completely different input mechanism.
        Since these are already published as release payloads,
        \texttt{ImportReleaseStep} is used instead.  It:
        \begin{itemize}
            \item
                imports the payload directly to the \texttt{release}
                \texttt{ImageStream} (via OpenShift)
            \item
                extracts the images to \texttt{stable*}, so that individual
                images can be used in the same way as integration streams
        \end{itemize}
    }
\end{frame}
